\documentclass[a4paper,11pt]{article}
\usepackage[margin=2cm]{geometry}
%\usepackage{anysize}
\usepackage[pdftex]{graphicx}
\usepackage{url}
\usepackage{fixltx2e}
\usepackage{listings}
\usepackage{textcomp}
\usepackage{wrapfig}
\usepackage{color}
\usepackage{xcolor}
\usepackage{framed}
\usepackage{subfig}
\usepackage{fancyhdr}
\usepackage{newclude}
\usepackage[nodayofweek]{datetime}
\usepackage[small,compact]{titlesec}
\usepackage[pdfborder=0]{hyperref}
\usepackage{verbatim}
\usepackage{lettrine}
\usepackage{enumitem}
\longdate

\setlength{\parskip}{11pt} 
\setlength\parindent{0pt}

\pagestyle{fancyplain}
\fancyhf{}
\lhead{\fancyplain{}{Machine Learning CBC}}
\rhead{\fancyplain{}{\today}}
\cfoot{\fancyfoot[R]{\thepage}}
%\fancyplain[R]{}{\thepage}}


\title{395 Machine Learning\\\Large{--- Assignment 2 ---}}
\author{Group 11\\Malon AZRIA, Alexandre CODACCIONI, Benjamin MAI,
  Laura HAGEGE.\\
  ma12917@ic.ac.uk, afc17@ic.ac.uk, bm2617@ic.ac.uk, lmh1417@ic.ac.uk \\
       \small{CBC helper: Pingchaun MA}\\
       \small{Course: CO395, Imperial College London}
}


\begin{document}
\maketitle

\section{Introduction}
    \include*{src/introduction}

\section{Q1: Explain the forward and backward pass of linear layers and relu activations}
    \include*{src/Q1}

\section{Q2: Explain the forward and backward pass of dropout}
    \include*{src/Q2}
 
\section{Q3: Explain the computation of softmax and its gradient}
    \include*{src/Q3}

\section{Q4: Training on the cifar10 dataset}
    \include*{src/Q4}

\section{Q5: Tuning the hyperparameters}
    \include*{src/Q5}
    
\section{Conclusion}
    \include*{src/conclusion}


\end{document}
