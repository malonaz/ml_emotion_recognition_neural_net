\begin{itemize}


\item \textbf{A1}:\\
  We cannot claim that one algorithm is a better learning algorithm than the other in general for multiple reasons.
  We will address the issue of using the classification error rate as a means of measure performances between models later.
  First we must address a more general issue with such a statement:
  different types of data and learning objectives context require different types of learning algorithms.
  Some algorithms are better suited for discrete classification, some other can output continuous data and some can even deal with 
  time-sensitive data such as speech recognition.
  Thus to say that a learning algorithm is better than another because of their respective performance on a single dataset is an uninformed mistake.\\
  Now that we have clarified this, let us move on to the performance measure used to compare the algorithms.
  Classification error rate is not enough.
  You must evaluate whether the model is good enough to solve your problem.
  What if your model is making decisions where you care more about it never being wrong, than about it being right?
  Suddenly, looking at a classification error rate is of little help.
  Instead, we would want to look at confusion matrices, precision and recall rates such that we may select a model with greater
  precision rate.
  Perhaps we would use an F-1 score weighted more heavily towards the precision rate.\\
  To conclude, one must carefully determine which performance measure to use when comparing models, and perhaps then, one may make
  a more specific claim on the performance of a model versus another, within the current context.
  
\item \textbf{A2}:\\

\end{itemize}
